\sectionbreak \section{ \standartTitleFont
  Теоретические сведения
}

\subsection{ \standartTitleFont
  Молоко и процесс пастеризации
}

{\standartFont

  \par Объектом исследования и внедрения нейронной сети является пастеризационная установка. Для понимая её технологического процесса, необходимо для начала понять с чем она работает и что она производит, а также какими функциями обладает пастеризационная установка. Поэтому в данной главе мы будем рассматривать пастеризационную установку как некий «серый ящик»: опишем объекты, которые попадают на вход и выходят из пастеризационной установки, а также опишем самые основные процессы, которые происходят с подаваемым на вход объектом.

  \par И так, основной объект обработки пастеризационной установки является молоко. Поэтому в дальнейшем молоко будет рассматриваться как объект технической обработки. Рассматривая его таким образом, мы понимаем, что оно должно обладать некоторыми показателями, например, состав молока, степень чистоты, кислотность, наличие токсичных и нейтрализующих веществ. При этом молоко обладает ещё и различными свойствами: органолептическими, физико-механическими и биохимическими. И так, разберём лишь те параметры и свойства, которые будут нам в дальнейшем интересны.

  \par Молоко можно разделить на две составляющих: вода и распределённые в этой воде пищевые вещества. К таким веществам относят жиры, белки, углеводы, ферменты, различные минеральные вещества и газы. Помимо этого, в молоке могут находится различные микроорганизмы. И как известно, некоторые из этих микроорганизмов, содержащиеся в молоке, являются опасными или вредными, например, бруцеллеза, ящура, возбудитель кишечной палочки и другие. Но как избавиться от вредных и опасных микроорганизмов?  Для этого используется процесс пастеризации {--} уничтожение различных форм вредных и опасных микроорганизмов в молоке. Но при этом молоко должно сохранить свою биологическую и питательную ценность, а также и своё качество.

  \par Однако перед тем, как перейти к пастеризации, необходимо определить, какое молоко можно пастеризовать, каким требованиям оно должно соответствовать. Для определения подходящего перед пастеризацией молока существует множество различных показателей и требований. Так, например, пригодное для пастеризации молоко должно быть кислотностью не более 22 °T, а бактериальная обсеменённость молока должна быть один миллион клеток на сантиметр кубический. При этом, молоко не должно быть вспененным. Перед пастеризацией, молоко должно быть также предварительно очищено на фильтрах или на сепараторах-молокоочистителях. Что ж, основные требования перечислены.

  \par А теперь к самой пастеризации. Основные параметры пастеризации есть температура пастеризации, а также время выдержки, то есть время нахождения молока в данном процессе. Относительно данных параметров существует выражение, выведенное Г. А. Куком и называемая критерием Пастера. Этот критерий можно рассчитать по формуле \ref{eq:PasterRule}.

	\begin{equation} \label{eq:PasterRule}
    P=\frac{t}{p}
  \end{equation}

  \begin{tabular}{p{0,3cm}p{15,975cm}}
		где  & $t$ {--} время действия температуры пастеризации, с; \\
		 	   & $p$ {--} время бактерицидного действия температуры пастеризации, с. \\
  \end{tabular}

  \par Что такое бактерицидное действие температуры пастеризации? Это, как раз и есть эффект, в результате которого происходит уничтожение вредных и опасных микроорганизмов.

  \par Также известна и ещё одна немаловажная зависимость: продолжительность выдержки зависит от температуры пастеризации. Зависимость показана в формуле \ref{eq:WidOtTemp}.

  \begin{equation} \label{eq:WidOtTemp}
	  \ln{t} = 36.84 - 048T 
	\end{equation}

  \par где $T$ {--} температура пастеризации, $^{\circ}$C.

  \par Завершение процесса пастеризации характеризуется полным уничтожение содержащихся в молоке микроорганизмов. Это можно будет определить благодаря уже известному критерию Пастера, значение которого должно быть не меньше единицы, для того чтобы считать, что процесс пастеризации завершён.

  \par Перед тем, как перейти к описанию пастеризационной установки, следует коротко разобрать ещё одно понятие, с которым мы будем сталкиваться в дальнейшем, а именно гомогенизация. Она представляет собой процесс дробления или уничтожения жировых шариков, образовавшихся в ходе хранения молока. Под воздействием внешних сил можно достичь значительного уменьшения объёма жировых шариков. Процесс гомогенизации позволяет предотвратить самопроизвольное отстаивание жира в молоке на производстве или при его хранении. При этом, гомогенизация даёт возможность сохранить однородную консистенцию молока. Далее рассматривать гомогенизацию так подробно, как процесс пастеризации, не имеет смысла, поскольку процесс гомогенизации не является центральным понятием предметной области.

  \par
}

\subsection{ \standartTitleFont
  Временные ряды и их прогнозирование
}

{\standartFont

  \par Одним из ключевых понятий данной работы является понятие временных рядов. Что же это? Временные ряды {--} это, по сути, некоторая последовательность, каждый элемент из которой состоит из двух или более параметров, а один из них обязательно должен обозначать время. Причём, все эти элементы в последовательности расположены в хронологическом порядке, то есть в порядке возрастания параметра времени.

  \par Параметр времени может быть представлен в разных форматах. Выбор формата времени зависит от задачи, удобства использования, длительности, в пределах которой будут собираться данные, а также от требуемой точности. Например, в случае если данные фиксируются раз в день, то хорошо подойдёт отсчёт времени по дням с указанием месяца и года. Если же данные фиксируются в определённые моменты дня, то к вышеописанному стоит прибавить указание часа и минуты фиксации. При необходимости можно указывать и секунды, и миллисекунды. Но что, если не особо-то и важно знать, в какой год, месяц или день это происходило, когда важно знать, сколько прошло времени от начала того или иного процесса? Тогда, скорее подойдёт формат дискретного времени. С помощью этого формата можно узнать длительность процесса в единственной выбранной нами единице измерения времени. Например, если сохранять время в секундах, то 1000-ча секунд сохранит свой формат 1000-чи секунд, время не будет переведено в 16-ать минут и 40-ок секунд. Всё это позволяет не привязываться к датам, которые не особо-то и влияют на технологический процесс промышленного оборудования.

  \par Остальные параметры могут уже характеризовать или описывать какой-либо процесс или какие-либо процессы, причём даже не обязательно одного элемента, а даже целой системы элементов. Так, например, когда элемент последовательности состоит из двух признаков, а один из которых, как уже известно, время, то второй, конечно же, уже будет обозначать характеристику или состояние изучаемого нами элемента. Но как только появляется три или более признаков, тогда уже можно говорить о фиксации характеристик или состояний разных элементов изучаемой системы в один и тот же момент времени или же о фиксации характеристики или состояния одного из множества элементов системы, но с указание этого элемента, например, с помощью идентификационного номера элемента. Как можно убедиться, временные ряды дают весьма гибкую возможность описания процессов или систем относительно времени.

  \par Технологический процесс пастеризационной установки как раз и представляет собой временной ряд, в котором имеется информация о показаниях различных датчиков в определённые моменты времени. Поэтому, говоря о данных технологического процесса пастеризационной установки, мы будем понимать, что они имеют форму временных рядов.

  \par А что из себя представляет работа с временными рядами? В основном работа делится на две части. Первая часть {--} это понимание структуры временного ряда, его закономерностей, таких как цикличность, тренд, сезонность и так далее, обработка данных временного ряда, визуализация данных, в общем, это всесторонний анализ временного ряда. Если опускать различные математические, статистические и тому подобные подробности, то анализ также может дать нам возможность понять, как начинался процесс, как шёл, развивался и на чём он закончился или остановился на данном моменте времени.  Строго говоря, нейронным сетям, как и исследователям временных рядов, тоже необходимо это понять, чтобы выполнить вторую часть работы, а именно, составление прогноза, что, зачастую, и является основной задачей работы с временным рядом. Да, анализ данных временного ряда технологического процесса делается с целью понять, что будет происходить с этим процессом дальше. Но чем нам так полезна информация о будущем, зачем она нужна? Для этого необходимо понять, что есть прогноз.

  \par Сам прогноз {--} это некоторая случайная величина, характеризующая вероятность того, что график в будущем пройдёт через определённую точку или некоторую область. Тогда прогнозирование {--} это получение максимально точных прогнозов, или, если говорить в отрыве от понятия прогноза, то это точное предсказание будущего, учитывающее исторические данные об объекте прогнозирования, а также знания о любых будущих событиях, которые могут повлиять на прогнозы.

  \par Как понять, что прогнозы действительные? Прогнозы являются таковыми, если они отражают подлинные закономерности и взаимосвязи, которые есть в исторических данных, при этом не повторяя прошлые события, которые более не актуальны или уже не повторяются.

  \par Что необходимо, чтобы составить хороший прогноз на основе временных рядов? Для этого обычно требуется выполнить следующие шаги:

  \par 1. Определить задачу.
  \par 2. Собрать информацию.
  \par 3. Произвести предварительный анализ.
  \par 4. Выбрать и создать модель прогнозирования.
  \par 5. Использовать и оценить модель прогнозирования.

  \par Вкратце разберём каждый пункт. При определении задачи необходимо понять, что вообще будет прогнозироваться, как будут использоваться прогнозы, благодаря чему будут получены прогнозы, кому эти прогнозы нужны или для чего. Второй пункт подразумевает непосредственный сбор или получение данных, а также оценка накопленного опыта людей, которые собирают данные и используют прогнозы. Для выполнения третьего пункта необходимо визуализировать данные, составить инфографику, если она необходима, определить взаимосвязь с признаками, закономерности временных рядов, качество данных и так далее, другими словами, провести анализ данных. На четвёртом пункте необходимо либо приобрести и адаптировать готовую модель, либо создать её самостоятельно. Но модель должна учитывать исторические данные, силы взаимосвязи между прогнозируемым признаком и любым другим признаком, а также способы использования прогнозов. В заключении проводится тестирование, оценивание, развёртывание и сопровождение модели прогнозирования.

  \par

}
